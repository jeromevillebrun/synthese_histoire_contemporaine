%Cette page a été concue pour le compilateur xelatex.
\documentclass[12pt]{report}
%Cette page a été concue pour le compilateur xelatex.

%Packages de base
\usepackage{xunicode} %pour l'unicode de XeLaTeX
\usepackage[a4paper]{geometry} %règle les dimensions de la page
\usepackage[french]{babel} %adapte aux spécificités de la langue française
\usepackage{hyperref} %pour des liens hypertexte
\usepackage{xurl} %pour avoir des url longs bien mis en page
\usepackage{enumitem} %nécessaire pour personnaliser la forme des listes

%Packages "en plus"
\usepackage{lettrine} %pour faire une première lettre géante
\usepackage{oldgerm} %pour une lettrine stylisée
\usepackage[babel]{csquotes} %commande \enquote{} un texte entre «“”»
\usepackage[final]{pdfpages} %pour ajouter page_de_garde_fial.pdf (\includepdf)
\usepackage{xspace}
\usepackage{numprint}

%Réglages
\setlist[itemize]{label=\textbullet} %utilise bullet point pour les listes
\renewcommand{\%}{\unskip\nobreakspace\char37\xspace} %espace insécable après le %


\begin{document}

\begin{titlepage}
\includepdf{page_de_garde_fial.pdf}
\end{titlepage}



%section = part
%partie = chapter
%chapitre = section
%sous-section = subsection
%minisection = subsubsection

\chapter*{Introduction}

\section*{Le concept de «~modernité~»}

Les spécialistes ne sont pas d’accord quant à l’origine de la modernité : pour les uns elle démarre dès le XII\up{e} siècle, pour d’autres à la Réforme ; mais tous s’accordent pour voir dans la Révolution de 1789 un tournant important. Il y a un consensus minimal entre les auteurs pour définir la modernité comme suit.

La modernité naît avec la réduction du «~pourquoi~» au «~comment~» qui ouvre à la \emph{rationalité scientifique}.\footnote{Le premier cri de quelqu'un qui se fait annoncer la mort d'un être cher est «~Pourquoi ?!~». Quand le scientifique arrive, il répond à la question en expliquant comment le camion en est venu à l'écraser, les mécanismes derrière tout cela, alors que ça n'était pas ça la question.} Il y a donc un déplacement de la demande de signification inhérente à toute société. En découle la \emph{pluralité} des vérités et la spécialisation des savoirs. On assiste alors non seulement à la séparation du temporel et du spirituel, mais aussi à la séparation des différents pouvoirs temporels. Ce qui a pour corollaire immédiat la dispersion de l’imaginaire. L’absence d’institution ayant le monopole du sens fait que la métaphorisation du religieux envahit différents domaines allant du stade de foot aux mises en scène du pouvoir.

En outre, cette rationalité implique un rapport nouveau à la vérité dont le doute ou les limites sont exclus.\footnote{La vérité est absolue, sans limite} Par la connaissance rationnelle sans limites, il s’agit de construire une image cohérente et totalisante du monde (porte ouverte au totalitarisme). C’est le thème de la maîtrise de l’univers par la connaissance ; une connaissance que rien ne borne et qui mène au progrès.\footnote{La connaissance d'aujourd'hui a des limites, mais en soi elle n'en a pas.} Ainsi, la modernité est marquée par la primauté du \emph{mouvement} sur l’ordre établi : l’avenir devient projet à réaliser pour être plus, être mieux ! Il en résulte que l’\emph{incertitude} devient une des grandes caractéristiques de la modernité\footnote{Aujourd'hui, la seule chose dont on est sûr, c'est que tout change.} ; ce qui exacerbe le besoin de sécurité individuel et collectif.\footnote{Aujourd'hui, on a des publicités qui jouent sur les deux : nouveauté et tradition.}

En effet, la question du «~pourquoi~» ---~rendue à l’individu par la modernité~--- ressurgit toujours au niveau existentiel. C'est la dernière caractéristique qui est aussi la plus importante. Elle correspond au besoin d’identification et au besoin de sécurité inhérent à la nature humaine et qui implique des modes de croire, des croyances pratiquées. Or cette question ontologique\footnote{Philosophie de l'être.} ressurgit d’autant plus fortement que la modernité est un discours de convocation de l’homme comme auteur de son destin. La valeur première y devient l’\emph{accomplissement du sujet} : la modernité pousse à l’autonomie individuelle, voire à l’individualisme et donc, en même temps, dissout le lien social. Si bien que la modernité creuse doublement un besoin de sécurité, de soumission à l'ordre établi (dont vont se nourrir les totalitarisme). Finalement, elle produit son contraire !
Ainsi la modernité sape constamment les structures de plausibilité de tous les systèmes de croyance et en exacerbe le besoin. Même si au début elle était encore fortement liée aux thématiques religieuses de l’accomplissement et du salut, via l’utopie. La crise de la modernité gît dans cette béance utopique, ces idéaux sapés de l’intérieur ---~bien plus que dans ses promesses non tenues.

\subsection*{Tuyau}

C'est le fil rouge du cours. Elle a plein de questions à l'examen sur ça.

\begin{itemize}
	\item En quoi les fascismes sont-ils des produits de la modernité ?
	\item Présentez la modernité à travers des exemples vus au cours.
\end{itemize}

\section*{Début de l'Époque contemporaine}

Quand l'Époque Contemporaine commence-t-elle ? C'est une question impossible. La question, en réalité est, comment les historiens ont débattu là-dessus.

\subsection*{1789 ?}

Faut-il prendre 1789 comme date de départ ? C'est l'usage dans l'historiographie française depuis la troisième république. C'est la gloire de la France, le début d'une révolution qui, de proche en proche, a bouleversé considérablement les conceptions et les structures de la société occidentale. Mais il y a des objections. 

\begin{itemize}
	\item D'un point de vue strictement Français, la nuit du 4 août 1789\footnote{Abandon des privilèges, et donc abolition du régime seigneurial.} est un moment mille fois plus important que la prise de la Bastille le 14 juillet.\footnote{Il y avait peu de gens, c'était une petite révolte contre l'arbitraire du roi, purement symbolique.}
	\item Ça n'a pratiquement rien changé à la situation des masses laborieuses urbaines, et encore pire, rurales.
	\item Si on quitte le contexte purement Français, 1789 n'a plus aucune signification.
\end{itemize}

Comment faut-il voir les choses ? La Révolution française fait en réalité partie d'un vaste mouvement révolutionnaire, la révolution Atlantique, qui va des années 1776 (États-Unis), et qui se prolonge avec les autres (1789, 1820, 1830 et le printemps des peuples en 1848). En 1770, on est encore dans l'Ancien Régime ; en 1848, on est dans la modernité, système libéral, capitaliste, d'économie de marché ; entre les deux, c'est une période de bascule, de grands bouleversements. Ils sont de tout ordre \footnote{Démographique, économique, social, politique, culturel.} et la France n'est qu'un maillon particulièrement important de cette chaîne.

En effet, tout ne s'est pas joué à la Révolution française. Même si elle se présente d'abord comme universelle\footnote{Car issue d'idées à vocation universelle comme les droits de l'homme.}, elle devient très vite un problème national français, à cause des menaces intérieures que sont les contre-révolutionnaires, lesquelles débouchent sur les guerres d'expansion dès 1792 (pas si universel, finalement). De plus, les aspirations démocratiques de 1789-1791 basculent assez rapidement dans l'accaparement bourgeois, doublé d'un impérialisme militaire, avec le directoire\footnote{1795.}, le consulat\footnote{Débute avec le coup d'état du 18 brumaire 1799.}, qui va mener finalement à l'Empire et au code napoléonien en 1804.


\subsection*{1848 ?}

Pourquoi ne pas prendre le bout de la chaîne, 1848, le Printemps des peuples ? Il y a de bons arguments.
\begin{itemize}
	\item Cette vague de révolutions agite une grande partie de l'Europe.
	\item Ces révolutions ont un caractère nouveau : elles ne sont plus seulement politiques et nationales, mais aussi sociales. On est passé d'une société d'ordres à une société de classes.\footnote{«~Ce fut la première grande bataille entre deux classes qui divisent la société moderne~» (Karl Marx à propos de l'insurrection populaire de juin 1848 en France).}
	\item En Autriche-Hongrie, elle marque la fin de l'Ancien Régime, avec l'abolition des droits féodaux.
\end{itemize}

Cependant, nous ne choisirons pas cette date parce que trop de pays ne sont pas concernés.
La Belgique\footnote{Elle a fait sa révolution en 1830, plus besoin de 1848.}, la Grande Bretagne (première puissance mondiale), les États-Unis et la Russie (deux pays qui vont dominer au XX\up{e} siècle), l'Amérique latine\footnote{La fin de l'ère coloniale de l'Amérique latine se situe dans les années 1820, avec les grands \emph{libertador} qui vont voter les constitutions} et les pays scandinaves.


\subsection*{1815}

Faut-il prendre la date de 1815 (congrès de Vienne) ? C'est bien celle que l'on a choisi, mais il y a des objections :
\begin{itemize}
\item À ce moment-là, l'Ancien Régime est loin d'avoir totalement disparu. Pire, il s'agit d'une restauration.\footnote{C'est une première restaurations. On restructure l'Europe dans une perspective traditionnelle, on remet les anciens monarques sur le trône, on efface la révolution Française et l'Empire. La deuxième restauration a lieu après 1848.}
\item Les révolutions de la période 1815-1848 font partie d'un cycle révolutionnaire. C'est un peu artificiel de couper au milieu.
\end{itemize}

Mais on peut justifier. Cette date a de grands avantages : 
\begin{itemize}
	\item Elle est valable pour l'ensemble de l'Europe. Cela permet de sortir de l'historiographie franco-centrée.
	\item Elle a aussi du sens au plan économique. C'est à cette Époque (1810-1820) que s'accentue le phénomène de la révolution industrielle, qui va bouleverser le paysage social et culturel (en plus de l'économique) de l'Europe.  
	\item De manière purement pratique, c'est la date que retiennent la plupart les manuels de synthèse scientifique  : «~L'Europe de 1815 à nos jours~». C'est donc plus pratique pour chercher dans les bibliographies.
\end{itemize}



\section*{Fin de l'Époque contemporaine}

Quand l'Époque contemporaine se termine-t-elle ? Théoriquement, cela ne pose pas de problème. L'époque contemporaine continue jusqu'à nos jours. Mais est-il réellement possible d'Écrire une histoire du temps présent qui réponde aux exigences scientifiques du travail d'historien, de la critique historique, de l'heuristique pour les 20 dernières années ? Il y a eu de grands débats là-dessus entre les historiens à la fin du 20\up{e} siècle.

\paragraph{Certains sont Contre.}
\begin{itemize}
	\item Difficulté d'avoir accès aux sources. Il y pour certains documents officiels ou privés, il y a des lois de 30 ans (voire 50 ou 100).
	\item Difficulté de dire ouvertement toute la vérité sur une question tant que les principaux acteurs sont encore en vie. C'est délicat aussi si leurs descendants sont encore en vie.
	\item L'historien, normalement, contrairement au journaliste, bénéficie du recul de l'histoire. Par exemple, le coronavirus, on en a parlé dans la presse, mais personne ne sait encore si ça marquera l'histoire ou si c'est assez insignifiant.
\end{itemize}

\paragraph{D'autres sont pour.}
\begin{itemize}
	\item La difficulté de s'affranchir du fait qu'on a vécu les évènements dont on parle n'est pas un problème puisqu'il n'est pas propre à l'histoire immédiate, ni à l'histoire contemporaine en général.
	\item Le problème de la difficulté d'accès aux documents est réel, mais moins grave qu'il n'y parrait. L'importance des documents officiels dans notre société diminue de plus en plus, d'une part parce que les auteurs écrèment les documents\footnote{Mitterand a écrémé ses archives avant de partir pour orienter le travail des historiens en sa faveur.}, d'autre part parce que de plus en plus de choses importantes se traitent par des moyens de communication qui ne laissent pas de trace (par téléphone ou de vive voie). 
		
		De plus, une série de documents officiels sont publiés plus vite qu'avant et sont mis sur la place publique au nom de la transparence démocratique dans le cadre de commissions parlementaires.\footnote{On nomme des commissions parlementaires d'historiens, par exemple sur l'affaire des biens juif, ou sur l'implication de la Belgique dans l'assassinat de Patrice Lubumba (non coupable, mais non-assistance à personne en danger car elle savait et elle a laissé faire).} À ces occasions-là, toute une série de documents officiels, qui n'auraient pu être publiés que quelques dizaines d'années plus tard, le sont tout de suite.

		Les mémoires et souvenirs publiés sont de plus en plus à la mode (hommes d'État, femmes d'hommes d'État, hommes politiques, diplomates, stars, maîtresses, etc.). Ces documents doivent être passés au crible de la critique, mais recoupés, ils représentent des informations intéressantes.

	\item Enfin, l'histoire du temps présent présente un grand avantage. On peut recourir aux sources orales, les seuls documents créés par les historiens. Elles nécessitent également d'être critiquées finement car elles ne donnent pas toujours des renseignements précis (la mémoire est ce qu'elle est), mais elles sont intéressantes pour retrouver l'atmosphère, l'ambiance. Elles permettent souvent de relativiser et de lire les documents d'archives avec en tête d'autres questions de recherche dont on n'aurait pas eu l'idée sans ça. Les médias, d'une part orientent l'opinion publique, et d'autre part la reflètent.
\end{itemize}

%Continuer à 11:25 (enregistrement 2)


\part{Le XIX\up{e} siècle}

\chapter{Les courants de la  pensée politique et sociale au XIX\up{e} siècle}

En introduction à l'histoire, on nous avait présenté 3 courants de pensée politique majeurs du 19\up{e} siècle : traditionalisme, libéralisme et marxisme. Ils restent les plus importants ! Tout ce qu'on va faire là, c'est simplement nuancer la caricature que l'on a fait en introduction à l'histoire.

\section*{Introduction : les concepts d'idéologie et d'utopie}

Le concept d'idéologie est assez vague. Il est utilisé dans un sens tantôt neutre (notre cas) voire laudatif, tantôt critique voire péjoratif. Il existe autant de définition du mot \emph{idéologie} que d'auteurs qui s'intéressent à ce concept. Quoi qu'il en soit, on va retenir deux définitions.

\begin{quote}
	«~L’idéologie est un système global d’interprétation du monde historico-politique~» (Raymond Aron) 
\end{quote}

Elle est donc inséparable des représentations mentales et collectives. En effet, on peut voir les représentations mentales et collectives comme un iceberg dont la partie immergée est l'ensemble des représentations (en partie) inconscientes, et dont la partie émergée est l'ensemble des représentations conscientes.

\begin{quote}
«~Une idéologie a pour fonction de donner des directives d’action individuelle et collective~» (Maxime Rodinson)
\end{quote}

L'idéologie n'est donc pas simplement du prêt-à-penser. C'est un projet qui débouche sur de l'action, ce qui mène à des luttes d'intérêts : «~Est-ce que c'est cette action-là qui va dans mon intérêt ?~». L'intérêt dans l'action va déterminer le choix de l'idéologie.


Ajoutons à cela l'intéressante réflexion du philosophe français Paul Ricoeur. Il voit idéologie et utopie simplement comme deux types de manières de penser la société (deux versans d'une même médaille, en somme). Il plaide pour un retour aux utopies car notre monde en manque, il est froid et a besoin d'utopies.

Il définit l'utopie comme un récit qui présente une image de l'avenir qui apparaît comme le contraire du présent dans ce qu'il a de malheureux. À la guerre, on oppose l'idéal d'une paix universelle. En rêvant l'impossible \footnote{Si on s'imagine que l'utopie est possible, on bascule dans le totalitarisme.}, elle a pour fonction de réaliser le possible. On sait que c'est impossible, mais on y va quand même.
Le fait que l'utopie soit opposée à la réalité du présent\footnote{Elle est un non-lieu, par étymologie.} éclaire cette réalité, qui apparaît, en comparaison, étrange. Plus rien n'est établi, le champ des possibles s'élargit et on envisage de nouvelles manières de vivre.\footnote{Un exemple d'utopie chez les chercheurs, est le \emph{slow science}. C'est le rêve de pouvoir ne publier qu'un seul travail parfaitement original chaque année, ce qui est impossible car ils sont de plus en plus évalués en terme de productivité, de nombre d'articles.}

En quoi utopie et idéologie sont-ils les remèdes l'un de l'autre ?
Dans tous les systèmes d'autorité, on observe toujours un fossé entre la prétention du pouvoir à la légitimité et la croyance en cette légitimité. Le rôle de l'utopie est de creuser ce fossé, le dénoncer, tandis que le rôle de l'idéologie est de le nouer en se structurant dessus : «~Nous voulons avoir le pouvoir pour faire un projet qui sera bon pour tous. Croyez-nous : cela supprimera le fossé entre la prétention du pouvoir à la légitimité et la croyance que vous avez en cette légitimité !~»


%#25:34.1#

\section{Le socialisme pré-marxiste ou utopiste}

\subsection*{Introduction : Rapports avec le Libéralisme (rappel)}

Pour rappel, c'est bien contre le libéralisme que les socialistes prémarxistes vont apparaître, ce libéralisme qui ce veut rationnel, qui promeut le dialogue et qui porte au pinacle l'individu qui doit se déployer dans toutes les dimensions (politique, social, moral, etc.). Les libéraux ont une méfiance vis-à-vis de l'État, qui doit intervenir le moins possible dans l'économie, méfiance vis-à-vis de l'Église (anticléricalisme), mais aussi méfiance vis-à-vis des masses. Ils ne veulent pas de la dictature des masses qui opprime l'individu. Cette mentalité-là est dominante ! C'est dans ce contexte-là que le socialiste pré-marxiste naît, et est qualifié d'utopiste par Marx (de manière un peu méprisante).

%Copié-collé de la synthèse de Pierre De Wael et Julien Delattre.

Avant tout, quelques remarques préliminaires.

Durant la première moitié du 19\up{e} siècle, il n'y a pas de différence nette entre le libéralisme radical\footnote{Branche du libéralisme la plus sensible aux revendications des masses} et le socialisme pré-marxiste. Au fur et à mesure de l'accaparement, le libéralisme radical a souhaité étendre le suffrage de plus en plus. 

À l'inverse, il ne faut pas confondre socialisme et catholicisme social, lequel nait au sein du traditionalisme avant de s'y opposer. Certains acteurs comme Philippe Buchez croient que l'Église ne doit pas être un bastion du conservatisme, mais qu'elle doit être fidèle à l'Évangile et revendiquer les libertés des droits de l'homme (l'Église doit être une force révolutionnaire).

Les doctrines les plus originales ne sont pas celles qui vont connaître le plus de succès. Par exemple, Louis Blanc sera beaucoup plus célèbre auprès des ouvriers que Proudhon ; or, ce dernier est un penseur qui est dix étages au-dessus de Louis Blanc. Il est beaucoup plus important au plan de la philosophie politique. Pourtant, c'est bien Louis Blanc, son successeur qui aura le plus d'impact.

Le socialisme pré-maxiste est aussi très lié à la révolution industrielle. Très vite, on s'est aperçu que l'optimisme des économistes libéraux (comme Adam Smith) ne se justifiait pas. Ils pensaient qu'avec la libre concurrence, tout allait s'équilibrer comme par magie, mais on s'est rapidement aperçu qu'elle ne produit pas l'égalité des conditions, mais, au contraire, la concentration des fortunes, qui entraine des crises jamais vues sous l'Ancien Régime, notamment en 46-48, crises de surproduction, crises industrielles, crises de financement. Enfin, le développement des grandes industries ne fait qu'aggraver le sort des ouvriers. À partir de ce constat-là, les socialistes pré-marxistes font alors une critique extrêmement dure des abus du machinisme. Leur critique est scientifiquement valable, mais les solutions qu'ils trouvent ne sont pas les bons. Ils cherchent encore dans le passé les bons remèdes. C'est là que Marx les traite d'utopistes : parce qu'il les trouve naifs.

\subsection{Projet global}

Tous les socialistes pré-marxistes, se veulent des réformateurs de la société, des organisateurs d'un monde nouveau, idéal, où tous les hommes accèderaient au bonheur, au lieu de vivre dans la misère comme c'est le cas. Le peuple est leur obsession.

Cette conception apparaît assez clairement chez Charles Fourrier\footnote{Les deux plus grands auteurs pré-marxistes sont Proudhon et Fourrier.}, qui a une interprétation bien à lui de l'histoire de l'humanité, avec des successions de phases. D'abord, une phase primitive, caractérisée par la sauvagerie et l'inexistence de la société\footnote{En réalité, même la société préhistorique était organisée.}. Puis, une phase patriarchale, organisée sur le modèle de la famille traditionnelle, société dominée par les mâles (patriarcat), qui correspond à l'Ancien Régime. Puis, la phase de la civilisation, phase adulte, qui est le monde actuel de Fourrier (premier 19\up{e} siècle), et qu'il trouve décevante. C'est pourquoi il est convaincu qu'il faut un changement radical de la société, qui devrait conduire à une nouvelle étape, un nouvel État social, dans laquelle la société serait plus humaine.
Comme utopiste, il rejette le pouvoir comme fin en soi.

\subsection{Les solutions}

Tous les utopistes sont d'accord avec ce que dit Charles Fourrier jusqu'ici. Là où ils se séparent, c'est pour trouver les solutions pour atteindre cette société. On peut distinguer trois voies : associative, technologique et politique.

\subsubsection{La voie associative}

Il y a deux groupes.
Pour les uns, ces associations sont des communautés modèles de vie. Pour les autres, ce sont des groupements de travailleurs.

Les communautés modèles, c'est d'abort l'idée de Robert Owen\footnote{Robert Owen (1771-1858), est le type de l'industriel anglais, philanthrope et protestant. Un grand bourgeois, mais qui a le soucis des plus petits.}, pour qui l'amélioration de la condition ouvrière va se faire à travers la formation de petites communautés agraires de type communistes. Des sortes de villages modèles dans lesquels toute propriété privée serait exclue, et tous les moyens de production seraient mis en commun.
C'est agraire : c'est encore un homme du 18\up{e}, il va chercher dans son imaginaire, où le bonheur est dans ces petits villages qui font du travail agricole.

Dans la même veine, il y a les phallanstères, imaginés par Charles Fourrier. Les phallanges sont formées de 1800 hommes, femmes et enfants, qui s'engagent de façon volontaire.
Là aussi, la critique du monde industriel va chercher des solutions dans le monde agricole.\footnote{Finalement, ils sont très proches des idées écologistes en vogue aujourd'hui.} 
Ici, on n'est pas dans un système communiste, parce que chacun est prélevé selon trois critères : ses capacités, le capital qu'il a apporté, et le talent qu'il a. pour fourrier, une égalité complète entre les individus n'est pas nécessaire. 
On met en commun, et chacun en retire ce qu'il a apporté.

%#9:57.3#

Un petit peu différente est l'idée de Louis Blanc, qui s'inspire de Proudhon, avec les groupements de travailleurs. Avec le mutuellisme proudhonnien, on n'est plus dans les communautés de vies. C'est le démarrage de l'idée de mutuelle, système d'échange au centre duquel les membres d'un groupe se garantissent réciproquement que, si quelqu'un apporte quelque chose, il reçoit la même chose (que ce soit service, crédit, information, valeur, bonne foi, vérité, liberté, propriété, etc.). Il s'agit, à la différence des phallanstères, d'un système égalitaire. Chaque membre, à condition de remplir les mêmes obligations, a les mêmes droits.

Ce système s'appuie sur la notion de contrat social de Rousseau, mais il n'y a pas besoin de communauté de vie. Pour Proudhon, le mutualisme doit plutôt régir les rapports entre les hommes de manière générale (entre patrons et ouvriers, acheteur et vendeur, emprunteur et prêteur). Les rapports entre ces gens sont matérialisés par la \emph{banque du peuple}.
Il part du principe que les intérêts perçus par les banques sont illégitimes.\footnote{Elles s'en mettent plein les poches, et c'est pas juste.} Ils pensent pouvoir apporter la \emph{justice sociale} (là où les traditionnaliste restaient dans la \emph{charité}).
Pour cela, il faut organiser une banque qui prête aux travailleurs sans percevoir d'intérêts autres que les frais de fonctionnement.
Cela permettra aux ouvriers d'échapper à la dépendance envers les patrons en créant leur propres entreprises.

%#13:43.4#

C'est de cette idée-là que Louis Blanc\footnote{Celui qui a eu plus d'impact.} a eu son idée des \emph{ateliers sociaux}.
Il s'agit d'ateliers, dans lesquels les ouvriers pourront acheter leurs instruments de travail. Ils doivent se constituer dans les principales branches de l'industrie.
Pour créer de tels ateliers, Blanc est réaliste et compte sur l'État\footnote{Contrairement à Proudhon, qui est un anarchiste et ne veux rien avoir à faire avec l'État.}. Il faut que l'État mette la mise de départ (il est le seul à disposer des fonds nécessaires). L'action de l'État doit se limiter à cette mise de départ. Après, il doit simplement surveiller que tout se passe bien, mais l'atelier doit être géré par les ouvriers.

Ce plan d'organisation du travail de Louis Blanc sera repris après la révolution de février 1848\footnote{Souvenons, la vague révolutionnaire de 1848, à Paris (et aussi à Vienne) : une première explosion en février 48, puis deuxième explosion en juin 48.} par le gouvernement provisoire, mais dans une optique assez différente des socialistes pré-marxistes.
En effet, lors de cette première révolution de 48, les bourgeois sont toujours à la maneuvre. Ils ont peur de ces masses en effervescence, et essaient donc de calmer le jeu. 
Ce sont plus des ateliers de \enquote{charité}, et tout est fait pour qu'ils ne puissent pas réellement rentrer en concurrence avec la véritable industrie.
Cela débouche sur la deuxième révolution de 1848 (celle de juin) : le peuple hurle que ces ateliers sociaux ont été dévoyés en \emph{rateliers sociaux}.
Cette fois, en juin 48, ce sont bien les masses qui sont aux commandes de la révolution, mais elles se font mater.

\subsubsection{La voie technologique}

La voie technocratique est incarnée par Saint-Simon\footnote{Claude-Henri de Rouvroy de Saint-Simon (1760--1825) est un grand noble qui était libéral au départ. Il est déçu par le libéralisme parce qu'il est contre l'individualisme. Il est aussi contre la violence.}. Ils veulent une économie qui s'organise d'elle-même, à l'abri des intervensions maladroites des pouvoirs politiques.
Évidemment, ils faut la réguler, mais surtout pas par ces pouvoirs politiques. Plutôt par des technocrates économiques.
Ils veulent donc vouloir la création de toute une série d'institutions économiques, lesquelles nous semblent aujourd'hui évidentes. Par exemple, les conseils d'industrie et du travail, les tribunaux d'industrie, les associations industrielles.\footnote{Toutes ces institutions existent aujourd'hui, mais avec le politique. Saint-Simon voulait, lui, supprimer le politique.}
Ces institutions économiques seraient le cadre de la société nouvelle.
Le projet prévoit que, quand ce système est établi, le pouvoir politique soit remis aux producteurs, au pouvoir économique\footnote{Industriels, banquiers, exploitants agricoles, artisans, ouvriers, et même les savants et les grands tenants de la culture.}.
\enquote{Alors, au gouvernement des personnes succèdera celui des choses.}

Pour justifier cela, il s'appuie sur une parabole. Imaginez qu'un gigantesque cataclysme tombe sur la France, tuant les 50 plus grands princes, les 50 plus grands prélats.
Au bout de l'épidémie, comment se porte la France ? Très bien : aucun problème.
Maintenant, imaginons que ce cataclysme atteint les 50 plus grands industriels, exploitations agricoles, inventeurs. Le lendemain, la France se réveille brisée.
Les politiques, les princes, les privilégiés sont des inutiles. On n'en a pas besoin.
Par contre, les travailleurs, industries, scientifiques, etc. Ceux-là on en a besoin, et c'est à eux qu'il faut donner le pouvoir.

Ce qui est nouveau, c'est qu'il ne valorise que le travail. Avec Proudhon, la valeur importante était la justice, pas le travail. Saint-Simon, lui, avait pris comme devise une des nombreuses phrases lapidaires de Saint Paul : \enquote{Celui qui ne travaille pas n'a pas besoin de manger}.
C'est très neuf, en contradiction avec la pensée de Benjamin Constant, qui justifiait le suffrage censitaire par le loisir\footnote{Les riches ont le \emph{loisir} (l'instruction) pour faire les bons choix car le temps qu'ils ne passent pas à travailler, ils le passent à étudier.}.

En quoi ce système est-il socialiste ? Essentiellement par ses conséquences. D'abord, tout le monde doit travailler : Saint-Simon a des mots durs contre tous les nobles qui vivent dans l'oisiveté. Ensuite, dans sa prise de position sur la propriété, qu'il veut réorganiser par l'État.\footnote{Dès 1814, il avait écrit \enquote{il n'y a point de changement dans l'ordre social sans un changement dans la propriété}. Mais il s'était contenté de réclamer une réorganisation de la propriété sous le contrôle de l'État. Ses disciples, par contre, iront plus loin, en préconisant la suppression de l'héritage, l'appropriation collective des moyens de production et la répartition des biens \enquote{à chacun selon ses capacités, à chaque capacité selon ses œuvres} (pas un système égalitaire).}

\subsubsection{La voie politique}

Tuyau ! C'est la voie la plus importante des trois proposée pour parvenir à une société égalitaire.
Beaucoup\footnote{Même parmi ceux qui promeuvent la voie associative ou technocratique.} pensent que le seul moyen efficace pour changer la société réside dans la conquête par les masses laborieuses des droits politiques et l'établissement d'un régime démocratique. La lutte pour le suffrage universel, pour un régime républicain démocratique. 

La plupart font confiance dans le bon sens de la nation pour instaurer un tel régime, et réprouvent la violence.
Au fond, ils sont plus réformistes que révolutionnaires, malgré la description terrifiante que les traditionnalistes et les libéraux doctrinaux font des masses après juin 1848.\footnote{L'image du \emph{péril rouge}, le couteau entre les dents.}

En Angleterre, les \emph{chartistes}\footnote{Partisans de la \emph{charte du peuple}. Le mouvement chartiste aura lieu de 1838 à 1848.} réclament l'abolition du sens électoral, le suffrage universel, le vote au scrutin secret, l'égalité des districts électoraux. Le cœur de leur programme est l'établissement d'une démocratie, par le régime électoral.
La grande originalité du mouvement chartise est qu'il est le seul à constituer réellement un mouvement ouvrier populaire à cette époque (avant 1848, il est le seul). Son action sera cependant de courte durée, ne dépassant pas 1848. Les raisons de ce déclin sont intérieures et extérieures. Extérieures : en 1848, les conservateurs britannniques voient ce qu'il se passe en Europe occidentale, et voient ça d'un mauvais œil. Une nouvelle pétition est envoyée par les chartistes à la chambre des communes (6 millions de signatures). Les conservateurs, inquiets, vérifient la pétition à la loupe, et on découvre que beaucoup de signatures sont fausses. Cela jette le discrédit sur le mouvement. Intérieures : bien que le chartisme soit authentiquement ouvrier, il n'a pas su élaborer une idéologie de classe, révolutionnaire.
Les réformes proposées visent à ce que l'ouvrier accède au statut de petit bourgeois (la petite propriété).
\footnote{C'est assez typique. Quand le mouvement n'est constitué que d'ouvriers, ces derniers ne rêvent que de devenir des petits bourgeois. Ce sont en général des bourgeois intellectuels qui pensent de manière plus radicale pour le peuple, et puis le peuple adopte leurs idées pour faire changer les choses.}

\subsection{La fin des utopistes et l'avènement des marxistes (1848 et 1871)}

%#34:17.7#

La fin des utopies a eu lieu eu lieu en deux coups, deux répressions. Les premières répressions qui s'abattent sur l'Europe ont lieu après le printemps des peuples.\footnote{Globalement, le printemps des peuples est un échec car il débouche sur une deuxième restauration.} Le deuxième évènement qui crée un désespoir profond est l'échec et la répression de la Commune de Paris (1871). Le socialisme pré-marxiste est mort. Désormais, la voie est ouverte à des doctrines beaucoup plus pures, mais aussi plus en phase avec la réalité du monde industriel. En effet, les utopistes n'étaient pas tout à fait en phase avec leur temps, et n'avaient pas intégré cette réalité industrielle (rappelons-le, ils allaient chercher la solution dans le monde agraire).

Les marxistes l'avaient intégré, eux, la réalité industrielle. D'ailleurs, à partrir de 1889\footnote{1889 : fondation de la Deuxième Internationale. (1864 : fondation de la Première).}, la doctrine officielle du socialisme international devient le marxisme.

En conclusion, on peut se demander si la faillite des socialistes pré-marxistes ne réside pas dans cette incapacité à dépasser un mentalité protocapitaliste\footnote{\enquote{Protocapitaliste} signifie qu'ils ne sont pas en dehors du monde capitaliste, mais qu'il ne sont pas non plus encore tout à fait dedans.}. Cette mentalité les aurait empêché de découvrir les moyens d'action efficaces en milieu industrialisé (ce que feront les marxistes plus tard).

Paradoxalement, ce sont ces faiblesses du socialisme pré-marxiste qui nous les rendent intéressantes pour aujourd'hui.
Dès les années 1970, on se réintéresse à ces auteurs. Le mouvement hippie conteste la déshumanisation de ce monde hyperindustrialisé, cette \emph{société de consommation}. On tente de réhumaniser avec de petites communautés, plutôt à la campagne, où l'on fait un peu de travail agricole (mais pas trop quand même), ou bien avec des petites communautés urbaines : des quartiers communautaires où chaque famille a sa maison, mais le jardin, immense, est commun.
C'est aussi dans ces années-là qu'on construit Louvain-la-Neuve. En effet, les kots sont de petites communautés autours du \emph{commu}. Cela vient directement des années 70.
Le questionnement écologiste, lui aussi, réinterroge ces auteurs.

\section{Le catholicisme social}

\subsection*{Introduction : Rapports avec le traditionnalisme (rappel)}

Autre grande force des 19\up{e}--20\up{e} siècles : les traditionnalistes et les ultramontains\footnote{Les ultramontains sont les catholiques qui sont pour le renforcement du pouvoir du Saint-Siège.}. Ils prônent une société pyramidale, fondée sur le modèle familial du paterfamilias calqué sur la
société d'Ancien Régime (monarque absolu de droit divin et fraternité des monarques). La société fait l’homme, il
faut tenir son rang, l’individu n’existe pas (à l’inverse du libéralisme et à l’image du
socialisme). On aura une bonne société si chacun reste à sa place (antipode des libéraux). Seuls
points sur lesquels libéraux et conservateurs peuvent s’accorder : le respect de la propriété et la
méfiance envers l'état.

C’est au sein du traditionalisme que va naître le catholicisme social. C'est un terme qui date des années
1890. Il est très équivoque car il désigne des réalités bien différentes, allant d’un paternalisme charitable à la
démocratie chrétienne. De plus, il n’y a pas une doctrine catholique sociale unifiée,
seulement des tendances, des mouvements, etc. Mais tout montre que les catholiques de ce
siècle ne sont pas restés sourd au problème sociaux du temps. Il y a au sein du monde catholique
dominé par les traditionalistes, des gens sensibles aux difficultés apportées par la révolution
industrielle.

%Monseigneur Ketteler (pour l'orthographe de son nom)
%Autres noms marqués au tableau :
%- Doutreloux
%- Sangnier
%- Kantsky

\subsection{Origines et évolution}

\subsection{L'Encyclique \emph{Rerum Novarum} (1891)}

\subsection{Le début du XX\up{e} siècle}

\section{Socialisme et marxisme après Marx}

\subsection{Interprétation générale du marxisme}

\subsubsection{Le révisionnisme de Bernstein}

\subsubsection{La réplique de l'orthodoxie marxiste}

\subsection{Modalités de la révolution socialiste (rappel)}

(elle a repris là juste après la pause.)

\subsection{Rôle et structure du parti socialiste}

\subsubsection{L'action politique légale}

\subsubsection{Le parti comme instrument révolutionnaire}

\section{L'anarchisme}

\subsection*{Introduction : Rapports avec la Belle Époque (rappel)}

\subsection{Facteurs de diffusion}

\subsection{Fondements théoriques}

\subsubsection{Opposition à l'État}

\subsubsection{Opposition à la propriété}

\subsubsection{Opposition à la religion}

\subsubsection{Anti-individualisme}


\chapter{États et révolutions au XIX\up{e} siècle}

\section{Unification italienne et unification allemande}

\subsection{L'Italie}

\subsubsection{Le \emph{Risorgimento} et le mouvement révolutionnaire de 1830 (Mazzini)}

\subsubsection{La vague révolutionnaire de 1848}

\subsubsection{Le royaume de Piémont-Sardaigne, moteur de l'unification (Cavour)}

\subsubsection{La question italienne sur la scène internationale (1850-1870)}

\subsection{L'Allemagne}

\subsubsection{La prospérité prussienne}

\subsubsection{Bismarck et l'unification par les guerres (les duchés, Sadowa, Sedan)}

\section{La Commune de Paris, 1871}

\subsection{Les origines lointaines}

\subsubsection{Transformation de Paris}

\subsubsection{Réveil de l'agitation ouvrière}

\subsection{Les origines prochaines}

\subsubsection{Chute du Second Empire}

\subsubsection{Divorce entre Paris et la province}

\subsection{Le déroulement}

\subsection{Les significations}


\chapter{Les développements économiques et sociaux au XIX\up{e} siècle}

\section{La première révolution industrielle, 1815-1871}

\subsection{Évolution de la conjoncture économique}

\subsubsection{La période 1820-1850}

\subsubsection{La période 1850-1870}

\subsection{Transformations sociales et stéréotypes sociaux}

\subsubsection{Le modèle bourgeois}

\subsubsection{La figure de l'ouvrier}

\section{La deuxième révolution industrielle, 1871-1914}

\subsection{Évolution de la conjoncture}

\subsubsection{La crise de 1873-1894}

\subsubsection{La reprise économique de la Belle Époque}

\subsection{Le Darwinisme social (rappel : scientisme)}

\subsection{Les difficiles conquêtes du monde ouvrier : le cas de la Belgique}

\subsubsection{Les grêves de 1886}

\subsubsection{La nécessité de légiférer}

\section{L'expansion coloniale}

\subsection{L'ampleur du mouvement}

\subsection{Mobiles de l'expansion coloniale}

\subsubsection{L'explication économique}

\subsubsection{L'explication politique}

\subsubsection{Les faits : la Tunisie et l'Égypte}




\part{Le XX\up{e} siècle}

\chapter{Le nationalisme}

\section{La conception germanique}

\section{La conception latine}

\section{Une évolution complexe : le culte de la Patrie}


\chapter{La Première Guerre mondiale}

\section*{Introduction (débats historiographiques)}

\subsection{Les entrées en guerre}

\subsection{Le consentement à la guerre longue : le cas de la Belgique}

\subsection{Le viol de la neutralité et l'invasion}

\subsection{Les massacres de civils}

\subsection{Le front}

\subsection{La Belgique occupée}

\section{Les mémoires de guerre (héroïsation et stigmatisation)}

\section{La démobilisation des esprits}

\chapter{Les transformations économiques et sociales}

\section{Sortir de la grande guerre et reconstruire}

\section{L'Occupation de la Ruhr, 1923}

\section{La crise de 1929 et ses conséquences (rappel)}

\chapter{Les totalitarismes (fascisme et communisme)}

\section{Le Fascisme italien}

\subsection{Les caractéristiques du système}

«~Tout dans l’Etat, rien contre l’Etat, rien en dehors de l’Etat~» (Mussolini)

«~Je prends l’homme au berceau et je ne le rends au Pape qu’après sa mort~» (idem)

\subsection{Bilan du fascisme}

\subsubsection{Au niveau démographique}

\subsubsection{Au niveau économique}

\subsubsection{Au niveau social}

\subsubsection{Au niveau religieux}

\subsubsection{Au niveau culturel}

\section{Le National-Socialisme allemand}

\subsection{Le parti national-socialiste des travailleurs allemands}

\subsection{Les caractéristiques du système nazi}

- L’idéologie est décrite dans Mein Kampf (1925-1926).
- L’Etat raciste est au service d’une politique de la race

\subsection{L'hitlérisme en pratique}

\subsubsection{Un pouvoir fasciste}

- Centralisation du pouvoir
- Mobilisation idéologique
- Instauration d’un régime corporatiste
- La remise en ordre de l’économie

\subsubsection{Un pouvoir raciste}

\section{Le Stalinisme}

\subsection{Le triomphe idéologique}

\subsection{La centralisation du pouvoir et les pratiques de terreur}



\chapter{La Deuxième Guerre mondiale}

\section{Les origines du conflit}

\subsection{L'ombre portée de la Grande Guerre}

\subsection{La politique internationale et militaire allemande}

\section{La guerre 1940-45}

\subsection{Les opérations militaires}

\subsection{Les occupations}

\subsection{Les résistances et les collaborations}

\section{sortir de la guerre}

\subsection{La remise en place de l'État (rappel)}

\subsection{Les répressions des collaborations}

\subsection{Les mémoires de guerre}


\chapter{La construction européenne}

\section*{Introduction : l'héritage de la guerre (mémoire et pardon)}

\section{La coopération européenne intergouvernementale}

\subsection{Les organisations à caractère économique}

\subsection{Une organisation à caractère politique}

\section{Vers l'intégration : de la CECA (1951) à Maastricht (1992)}

\subsection{Les Utopies fondatrices (1940-1947)}

\subsection{La mise en route de l'Europe des six}

\subsection{La parenthèse De Gaulle (1963-1969)}

\subsection{L'élargissement de la CEE}

\subsection{Quelques critiques adressées à la CE}


\chapter{Les décolonisations}

\section*{Introduction}

\section{Les causes de la décolonisation}

\subsection{Les sources de mécontentement}

\subsection{Les conséquences de la Deuxième guerre mondiale}

\subsection{Un environnement international favorable}

\section{La décolonisation de l'Asie}

\subsection{Les métropoles face à la décolonisation}

\subsection{La décolonisation négociée}

\subsection{La décolonisation violente}

\section{La décolonisation de l'Afrique}

\subsection{L'Afrique du Nord}

\subsection{L'Afrique noire}


\part*{Conclusion}



\end{document}
