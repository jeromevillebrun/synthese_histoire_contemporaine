%Cette page a été concue pour le compilateur xelatex.

%Packages de base
\usepackage{xunicode} %pour l'unicode de XeLaTeX
\usepackage[a4paper]{geometry} %règle les dimensions de la page
\usepackage[french]{babel} %adapte aux spécificités de la langue française
\usepackage{hyperref} %pour des liens hypertexte
\usepackage{xurl} %pour avoir des url longs bien mis en page
\usepackage{enumitem} %nécessaire pour personnaliser la forme des listes

%Packages "en plus"
\usepackage{lettrine} %pour faire une première lettre géante
\usepackage{oldgerm} %pour une lettrine stylisée
\usepackage[babel]{csquotes} %commande \enquote{} un texte entre «“”»
\usepackage[final]{pdfpages} %pour ajouter page_de_garde_fial.pdf (\includepdf)
\usepackage{xspace}
\usepackage{numprint}

%Réglages
\setlist[itemize]{label=\textbullet} %utilise bullet point pour les listes
\renewcommand{\%}{\unskip\nobreakspace\char37\xspace} %espace insécable après le %

